\section{Cronograma de atividades} % (fold)
\label{sec:cronograma_de_atividades}

Exemplo de citação \cite{thepdf}\\
Exemplo de citação de site \cite{pinkybug}
% section cronograma_de_atividades (end)


\section{Metas de usabilidades} % (fold)
\label{sec:metas_de_usabilidades}
	A usabilidade foca proporcionar que produtos interativos sejam intuitivos e agradáveis de se usar. E para se atingir essa meta há um conjunto de objetivos a se cumprir. Segue eles: eficácia, eficiência, segurança, utilidade, fácil de aprender e fácil de lembrar o uso.

	Eficácia refere-se ao quanto o produto é bom para ser útil ao que ele foi concebido a fazer. Normalmente produtos cheio de ações a se fazer para se chegar a um objetivo não são eficientes ao uso.

	Eficiência está vinculado a maneira de como o produto auxilia a resolver um determinado problema.Ele pode ser fácil de usar mas talvez não seja de grande impacto.

	Segurança refere-se ao produto proteger o usuário de situações de dano e risco como, por exemplo, ele deletar seus arquivos pessoais por engano.

	Utilidade refere-se a capacidade do produto poder oferecer o tipo mais conveniente de funcionalidade com o intuito de se resolver um problema. 

	Facilidade de aprendizado está relacionado ao quanto o produto é fácil de aprender a usar.

	Facilidade de lembrar do uso está associado ao quanto o produto é intuitivo o suficiente ao ponto de ser intuitivo tambem lembrar do seu uso.

\section{Metas decorrentes da experiência do usuário} % (fold)
\label{sub:metas_decorrentes_da_experi_ncia_do_usu_rio}
	As metas da experiência de usuário é o produto final do design de interação constituído de metas de usabilidade alcançadas. 

	Essas metas estão no campo da subjetividade do usário. São valores subjetivos que o usuário percebe no produto. De certa forma através desses valores o usuário pode ter um envolvimento emocional com o produto em questão. 

	Envolver o usuário ao máximo a atividade do produto é um sinal de que a experiência está sendo positiva. Um sinal de que a tecnologia inicialmente de conhecimentos matematicos e físicos gera uma resposta emotiva do usuário. 

	Nem sempre as metas decorrentes da experiência do usuário e metas de usabilidade são convergentes. Cada um irá variar de acordo com o objetivo do produto a ser projetado.   

% subsection metas_decorrentes_da_experi_ncia_do_usu_rio (end)

% section metas_de_usabilidades (end)

%este é "se possivel"
\section{Story Board} % (fold)
\label{sec:story_board}

% section story_board (end)

\section{Estudo de questionários} % (fold)
\label{sec:estudo_de_question_rios}

% section estudo_de_question_rios (end)


\section{Ferramentas utilizadas} % (fold)
\label{sec:ferramentas_utilizadas}

% section ferramentas_utilizadas (end)
