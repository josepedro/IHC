\newglossaryentry{microcontrolador}
{
  name=Microcontrolador,
  text=microcontrolador,
  description={É um computador em chip, contendo um processador, memória e periféricos de entrada/saída. É um microprocessador que pode ser programado para funções específicas, em contraste com outros microprocessadores de propósito geral (como os utilizados nos PCs). Eles são embarcados no interior de algum outro dispositivo (geralmente um produto comercializado) para que possam controlar as funções ou ações do produto. Um outro nome para o microcontrolador, portanto, é controlador embutido}
}

\newglossaryentry{Linux}
{
  name=Linux,
  description={is a generic term referring to the family of Unix-like
               computer operating systems that use the Linux kernel},
  plural=Linuces
}

\newglossaryentry{portal}
{
  name=Portal,
  text=portal,
  description={Também chamado de Home Page. É o site em si}
}

\newglossaryentry{autonoma}
{
  name=Autônoma,
  text=aut\^onoma,
  description={Independente, tem autonomia e não necessita de terceiros para existir/funcionar}
}

\newglossaryentry{Wordpress}
{
  name=Wordpress,
  description={É um aplicativo de sistema de gerenciamento de conteúdo para web, escrito em PHP com banco de dados MySQL, voltado principalmente para a criação de blogs via web}
}

\newglossaryentry{plugin}
{
  name=Plugin,
  text=plugin,
  description={É um programa de computador usado para adicionar funções a outros programas maiores, provendo alguma funcionalidade especial ou muito específica. Geralmente pequeno e leve, é usado somente sob demanda}
}


%rede
%templates
%acesso administrativo

